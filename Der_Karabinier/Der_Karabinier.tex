\documentclass{article}
\usepackage{amsmath}

\begin{document}

\section*{Analogie tra Telegram e Meta Inc.}

Paga attenzione: 
\begin{itemize}
    \item Il primo è un virus: non cliccare.
    \item Meta è ok: clicca qui per chattare con Meta, donne calde aspettano.
    \item Dal prodotto di convoluzione fra tuo padre e una scimmia è nato di nuovo tuo padre ma com'è possibile.
\end{itemize}

\section*{Analogie fra equazioni di stato}

\subsection*{Sistema (DC) Donne Calde}

Donne calde sono di questo tipo:
\[
\alpha_1 \dot y_{(t)} + \alpha_2 y_{(t)} = h_{(t)}
\]
\\La rapidità con cui esse scambiano calore è proporzionale alla quantità $\sigma = \frac{\alpha_1}{\alpha_2}$.
$h_{(t)}$ è tuo padre \- la forzante \- che com'è detto sopra è una funzione generata dal prodotto di convoluzione fra sé ed una scimmia:
Interesssante è capire fin da subito cosa implichi questo risultato: supponiamo che tuo padre sia della forma $h_{(t)} =  e^{i \omega t}$.
\\
La forzante $h(t)$ è tuo padre, che come detto sopra è una funzione generata dal prodotto di convoluzione fra sé ed una scimmia.\\
Interessante fin da subito è capire le implicazioni di questo lemma.\\
Supponiamo che tuo padre \- e quindi i successivi risultati \- sia della forma $h^{n}_{(t)} = \Omega_n e^{i \omega t}$.\\
Tuo padre è detto \textit{autofunzione} della convoluzione con qualsiasi scimmia $s_{(t)}$ poiché la convoluzione ne preserva l'oscillazione complessa con frequenza $i \omega$ e altera l'ampiezza $\Omega_j = \lambda_{jk} \Omega_k$, con $\lambda_{jk} \in {R}$.

\subsection*{Circuito (RC) resistore-condensatore}

Per un circuito RC, l'equazione diventa:

    \[
    R \dot{q}+\frac{q}{C} = 0
    \]
dove:
    \begin{itemize}
        \item $R$ è la resistenza,
        \item $C$ è la capacità,
        \item $q = q_{(t)}$ è la carica sul condensatore.
    \end{itemize}
La soluzione di questa equazione è:
\[
q_{(t)} = q_0 e^{-\frac{t}{RC}}
\]
dove $q_0$ è la carica iniziale.

\subsection*{Sistema (m$\zeta$) massa-smorzatore}

Per un sistema $m \zeta$, l'equazione diventa:
\[
\zeta\dot{x}+kx = 0
\]
dove:
\begin{itemize}
    \item $\zeta$ è il coefficiente di smorzamento,
    \item $k$ è la costante elastica della molla,
    \item $x = x_{(t)}$ è lo spostamento della massa.
\end{itemize}

La soluzione di questa equazione è:
\[
x_{(t)} = x_0 e^{-\frac{\zeta}{m} t}
\]
dove $x_0$ è lo spostamento iniziale.

\subsection*{Sintesi}

\begin{itemize}
    \item \textbf{Circuito $R C$}: La carica sul condensatore si scarica esponenzialmente con una costante di tempo $\tau = RC$.
    \item \textbf{Sistema $m \zeta$}: La velocità del sistema decresce esponenzialmente con una costante di tempo $\tau = \frac{m}{\zeta}$. 
\end{itemize}

\end{document}
